\phantomsection
\addcontentsline{toc}{section}{Abstract}
\section*{Abstract}
The overall goal of this project was to create a device capable of generating detailed maps and imagery of an area in real-time. This device relied on an FPGA, and used image processing algorithms capable of determining the relative distances of objects surrounding the device. The device gathered camera imagery, as well as localization data and distance measurements from an IMU and a rangefinder.  A major deliverable of this project was that all output data from the device has been fully processed, allowing for the option of viewing it without further processing by the user.
\par
This device would be especially useful for first responders. It was intended to be mounted on a small remote control vehicle, allowing any connected user to wirelessly traverse dangerous and remote locations in search of people in need. Since this device could be capable of transmitting data in real-time, it would be able to provide first responders with an accurate representation of not only a 2D floorplan of an area such as a building, but also 3D localization data on its entire field of view. An anticipated use of this device would be in the event of a building in danger of collapsing. Since it would be dangerous to physically enter the building, first responders could locate any people trapped inside and find the fastest route to them using the wirelessly transmitted floorplan. The first responders would also be aware of any dangers in their way by making use of the real-time augmented video stream. This video stream would consist of image data with overlaid with a 3D depth map created by the image processing algorithms.